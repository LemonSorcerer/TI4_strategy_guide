{\handel\Huge The Ghosts of Creuss} \hfill {\Large Difficulty: \hard} \vspace{-4pt}\\
\hrule width \hsize \kern 1mm \hrule width \hsize height 2pt


\begin{multicols}{2}

\stress{Faction Abilities}

\begin{itemize}
\item \name{Quantum Entanglement}\\
You treat all systems that contain either an alpha or beta wormhole as adjacent to each other. Game effects cannot prevent you from using this ability.
\item \name{Slipstream}\\
During your tactical actions, apply +1 to the move value of each of your ships that starts its movement in your home system or in a system that contains either an alpha or beta wormhole.
\item \name{Creuss Gate}\\
When you create the game board, place the Creuss Gate (tile 17) where your home system would normally be placed. The Creuss Gate system is not a home system. Then place your home system (tile 51) in your play area.
\end{itemize}


\vspace{-10pt}\rule{\hsize}{0.4pt}\vspace{5pt}

\stress{Starting Units}

\vspace{-5pt}
\begin{multicols}{2}
\begin{itemize}
\item 1 carrier
\item 2 destroyers
\item 2 fighters
\item 4 infantry
\item 1 space dock
\color{white}
\item 1 PDS
\color{black}
\end{itemize}
\end{multicols}

\vspace{-5pt}
\stress{Commodities}: 4

\vspace{2pt}
\stress{Home System}: Creuss (4/2); \emph{Delta Wormhole}

\rule{\hsize}{0.4pt}\vspace{5pt}

\stress{Starting Technology}

\begin{itemize}
\item \gravdrive
\end{itemize}

\vspace{-10pt}\rule{\hsize}{0.4pt}\vspace{5pt}

\stress{Flagship}

\begin{itemize}
\item \name{Hil Colish}\\
Cost 8 | Combat 5 | Move 1 | Capacity 3 \\
\emph{Sustain Damage}\\
This ship's system contains a delta wormhole. During movement, this ship may move before or after your other ships.
\end{itemize}

\vspace{-10pt}\rule{\hsize}{0.4pt}\vspace{5pt}

\stress{Mech}

\begin{itemize}
\item \name{The Icarus Drive}
\\
Cost 2 | Combat 6 \\
\emph{Sustain Damage}\\
After any player activates a system, you may remove this unit from the game board to place or move a Creuss wormhole token into this system. 
\end{itemize}

\vspace{-10pt}\rule{\hsize}{0.4pt}\vspace{5pt}

\nounits

%\rule{\hsize}{0.4pt}\vspace{5pt}
\columnbreak
\stress{Faction Technologies}

\begin{itemize}
\item \wormholeGenerator
\item \splicer
\end{itemize}

\vspace{-10pt}\rule{\hsize}{0.4pt}\vspace{5pt}

\stress{Leaders}

\begin{itemize}
\item \name{Agent - Emissary Taivra}\\
\emph{After a player activates a system that contains a non-delta wormhole:}
\\
You may exhaust this card; if you do, that system is adjacent to all other systems that contain a wormhole during this tactical action.
\item \name{Commander - Sai Seravus} \textcolor{gray}{(\emph{Unlock:} Have units in 3 systems that contain alpha or beta wormholes.)}\\
After your ship moves, for each ship that has a capacity value and moved through 1 or more wormholes, you may place 1 fighter from your reinforcements with that ship if you have unused capacity in the active system. 
\item \name{Hero - Unknown} \textcolor{gray}{(\emph{Unlock:} Have 3 scored objectives)}\\
\emph{Action:} Swap the positions of any two systems that contain wormholes or your units other than the Creuss system and the wormhole nexus.
\end{itemize}

\vspace{-10pt}\rule{\hsize}{0.4pt}\vspace{5pt}

\stress{Promissory Note}

\begin{itemize}
\item \name{Creuss Iff}\\
\emph{At the start of your turn during the action phase:}\\
Place or move a Creuss wormhole token into either a system that contains a planet you control or a non-home system that does not contain another player's ships.\\
Then, return this card to the Creuss player.
\end{itemize}

\end{multicols}



